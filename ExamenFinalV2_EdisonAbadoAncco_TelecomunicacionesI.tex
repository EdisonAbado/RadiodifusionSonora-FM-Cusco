%% Estructura principal para un reporte de Trabajos intersemanales CIRCAE %%
\documentclass[a4paper]{IEEEtran} %tamaño del papel y el tipo de transcripción que será IEEE
%\usepackage[total={6.5in,10in},left=1in,top=0.5in,includehead,includefoot]{geometry}
\usepackage[utf8]{inputenc} %el tipo de codificación que incluye símbolos como la tilde
\usepackage[spanish]{babel} % hacemos que nuestro documentación vaya en español
\usepackage{cite} % citas bibliográficas
\usepackage{graphicx} %gráficos, usaremos solo .jpg o .png con estándares que ya veremos
\usepackage{subfigure} %usar subfiguras
\usepackage{url} %agregar direcciones url
\usepackage{amsmath} %expresiones matemáticas
\newtheorem{teor}{Teorema}[section] %definimos la enumeración de Teoremas usando la etiqueta \begin{teor} ... \end{teor} para los ejemplos, podemos darle etiquetas para referenciarlas a lo largo del texto
\newtheorem{ejem}{Ejemplo}[section] %definimos la enumeración de Ejemplos usando la etiqueta \begin{ejem} ... \end{ejem} para los ejemplos, podemos darle etiquetas para referenciarlas a lo largo del texto
\newtheorem{exper}{Experimento}[section] %definimos la enumeración de Ejemplos usando la etiqueta \begin{exper} ... \end{exper} para los Experimentos, podemos darle etiquetas para referenciarlas a lo largo del texto
\usepackage{setspace} %LA usamos para asignar el interlineado
%%%%%%% settings para incluir codigo fuente en cualquier lenguaje
\usepackage{listings} %comenzamos la configuración de nuestras lineas de codigo que se incluirá de ser necesario en el documento
\usepackage[usenames]{color} %seteamos el uso de nombre y color
\definecolor{gray97}{gray}{.97}%definimos nombre y color
\usepackage{textcomp}
\lstset{
	frame=Ltb,
	framerule=1pt,
	framextopmargin=5pt, %margen de arriba
	framexbottommargin=5pt, %margen de abajo
	framexleftmargin= 2pt, %separacion del margen izquierdo
	framesep=5pt,
	rulesep=0.3pt,
	backgroundcolor=\color{gray97},
	rulesepcolor=,
	tabsize=2,
	rulecolor=\color[RGB]{106, 182, 217}, %AZUL
	upquote=true,
	aboveskip={2\baselineskip}, %despues de la linea de texto
	columns=fixed,
	showstringspaces=false,
	extendedchars=true,
	breaklines=true,
	prebreak = \raisebox{0ex}[0ex][0ex]{\ensuremath{\hookleftarrow}},
	showtabs=false,
	showspaces=false,
	showstringspaces=false,
	basicstyle=\scriptsize\ttfamily\color[RGB]{39, 100, 46}, %Numeros de lineas, simbolos, puntos y coma y demas
	identifierstyle=\ttfamily\color[RGB]{56, 140, 189}, %variables
	commentstyle=\color[RGB]{62, 179, 101}, %comentarios
	stringstyle=\color[RGB]{247, 165, 42}, %impresiones
	keywordstyle=\bfseries\color[RGB]{237, 118, 150}, %funciones
	%
	numbers=left,
	numbersep=1pt, %separacion del numero
	numberstyle=\tiny,
	numberfirstline = false,
	breaklines=true,
}
\usepackage{graphicx}
\usepackage[colorinlistoftodos]{todonotes}
\usepackage{enumitem}
%%%%%%%
\providecommand{\keywords}[1]{\textbf{\textit{Términos Clave---}} #1}

\begin{document}
	%	\spacing{0.9} %definimos un interlineado de 0.9 para todo el documento
	
	\title{Radiodifusión sonora FM en la región de Cusco}
	\author{ING. FERNANDO TAGLE CARBAJAL, DOCENTE, EDISON ABADO ANCCO, 145012, ALUMNO
	}
	
	\markboth{UNIVERSIDAD NACIONAL DE SAN ANTONIO ABAD DEL CUSCO - INGENIERÍA ELECTRÓNICA - TELECOMUNICACIONES I - 2021 I -G6-P1-007}{} % Codigo del informe que corresponde a: - numero de grupo con la G antepuesta - numero de proyecto con la P antepuesta | número de informe
	\maketitle
	
	
\section{Acomayo - Acomayo}

\begin{itemize}
	\item Plan Aprobado por RVM Nº 426-2014-MTC/03 (28/07/2014)
	\item Con Restricción RD Nº 1676-2017-MTC/28 (18/08/2017)
\end{itemize}

El Registro Nacional de Frecuencias incluye como "autorizadas" a las estaciones con autorización vigente o que se encuentran en proceso de renovación, y no cuentan con resolución que la deje sin efecto o extinción firme. El estado “cautelar” corresponde a estaciones cuya autorización ha sido cancelada en la vía administrativa, pero cuentan con una medida cautelar otorgada por el Poder Judicia.

\subsection{Número de estaciones de Transmisión FM}

\begin{tabular}{|l|l|} \hline
	Canalización 			& 20 \\ \hline
	Autorizaciones			& 16 \\ \hline
	Cautelar				& 0 \\ \hline
	Reservado Estado		& 1 \\ \hline
	Disponibilidad			& 3 \\ \hline
	Trámite					& 2 \\ \hline 
\end{tabular}

\subsection{Límite de potencia de transmisión}

La Potencia Efectiva Radiada máxima en la dirección de máxima ganancia de antena a ser autorizada en esta localidad será: 0.5 KW.

\subsection{Límites geográficos de autorización}

La transmisioón está autorizada Para La provincia de Acomayo, y los distritos:

\begin{itemize}
	\item Acomayo 
	\item Acopia 
	\item Pomacanchi 
	\item Sangarará
\end{itemize}



\section{Anta - Anta}

\begin{itemize}
	\item Plan Aprobado por RVM Nº 108-2004-MTC/03 (10/07/2004)
	\item Con Restricción RD Nº 2189-2005-MTC/17 (12/01/2006)
	\item Concurso Público Nº 02-2012-MTC/28 (17/11/2012), RD Nº 1555-2012-MTC/28 (07/11/2012)
\end{itemize}

El Registro Nacional de Frecuencias incluye como "autorizadas" a las estaciones con autorización vigente o que se encuentran en proceso de renovación, y no cuentan con resolución que la deje sin efecto o extinción firme. El estado “cautelar” corresponde a estaciones cuya autorización ha sido cancelada en la vía administrativa, pero cuentan con una medida cautelar otorgada por el Poder Judicia.

\subsection{Número de estaciones de Transmisión FM}

\begin{tabular}{|l|l|} \hline
	Canalización 			& 20 \\ \hline
	Autorizaciones			& 24 \\ \hline
	Cautelar				& 0 \\ \hline
	Reservado Estado		& 2 \\ \hline
	Disponibilidad			& 2 \\ \hline
	Trámite					& 1 \\ \hline 
\end{tabular}

\subsection{Límite de potencia de transmisión}

La Potencia Efectiva Radiada máxima en la dirección de máxima ganancia de antena a ser autorizada en esta localidad será: 0.5 KW.

\subsection{Límites geográficos de autorización}

La transmisioón está autorizada para llegar hasta la provincia de Anta y Urubamba, los distritos/centro poblado:

\begin{itemize}
	\item Ancahuasi
	\item Anta 
	\item Huarocondo 
	\item Pucyura
	\item Zurite
	\item Chinchero
	\item Huayllabamba
	\item Maras
	\item Ollantaytambo
	\item Urubamba
	\item Yucay
\end{itemize}




\section{Calca - Calca}

\begin{itemize}
	\item Plan Aprobado por RVM Nº 108-2004-MTC/03 (10/07/2004)
	\item Con Restricción RD Nº 067-2011-MTC/28 (29/01/2011)
	\item Concurso Público Nº CP 02-2013-MTC/28 (18/12/2013), RD Nº 1841-2013-MTC/28 (12/12/2013)
\end{itemize}

El Registro Nacional de Frecuencias incluye como "autorizadas" a las estaciones con autorización vigente o que se encuentran en proceso de renovación, y no cuentan con resolución que la deje sin efecto o extinción firme. El estado “cautelar” corresponde a estaciones cuya autorización ha sido cancelada en la vía administrativa, pero cuentan con una medida cautelar otorgada por el Poder Judicia.

\subsection{Número de estaciones de Transmisión FM}

\begin{tabular}{|l|l|} \hline
	Canalización 			& 14 \\ \hline
	Autorizaciones			& 12 \\ \hline
	Cautelar				& 0 \\ \hline
	Reservado Estado		& 1 \\ \hline
	Disponibilidad			& 1 \\ \hline
	Trámite					& 1 \\ \hline 
\end{tabular}

\subsection{Límite de potencia de transmisión}

La Potencia Efectiva Radiada máxima en la dirección de máxima ganancia de antena a ser autorizada en esta localidad será: 0.5 KW.

\subsection{Límites geográficos de autorización}

La transmisioón está autorizada para llegar hasta la provincia de Calca, los distritos/centro poblado:

\begin{itemize}
	\item Calca
	\item Calca - Totora
	\item Coya
	\item Lamay
\end{itemize}


\section{Canas - Yanaoca}

\begin{itemize}
	\item Plan Aprobado por RVM Nº 108-2004-MTC/03 (10/07/2004)
	\item Con Restricción RD Nº 2908-2017-MTC/28 (19/01/2018)
\end{itemize}

El Registro Nacional de Frecuencias incluye como "autorizadas" a las estaciones con autorización vigente o que se encuentran en proceso de renovación, y no cuentan con resolución que la deje sin efecto o extinción firme. El estado “cautelar” corresponde a estaciones cuya autorización ha sido cancelada en la vía administrativa, pero cuentan con una medida cautelar otorgada por el Poder Judicia.

\subsection{Número de estaciones de Transmisión FM}

\begin{tabular}{|l|l|} \hline
	Canalización 			& 15 \\ \hline
	Autorizaciones			& 10 \\ \hline
	Cautelar				& 0 \\ \hline
	Reservado Estado		& 1 \\ \hline
	Disponibilidad			& 4 \\ \hline
	Trámite					& 1 \\ \hline 
\end{tabular}

\subsection{Límite de potencia de transmisión}

La Potencia Efectiva Radiada máxima en la dirección de máxima ganancia de antena a ser autorizada en esta localidad será: 0.5 KW.

\subsection{Límites geográficos de autorización}

La transmisioón está autorizada para llegar hasta la provincia de Canas, el distrito/centro poblado:

\begin{itemize}
	\item Yanaoca
\end{itemize}



\section{Canchis - Sicuani}

\begin{itemize}
	\item Plan Aprobado por RVM Nº 108-2004-MTC/03 (10/07/2004)
	\item Con Restricción RD Nº 1806-2006-MTC/17 (02/09/2006)
	\item Concurso Público Nº 01-2020-MTC/28 (30/12/2020), RD Nº 1891-2020-MTC/28 (30/12/2020)
\end{itemize}

El Registro Nacional de Frecuencias incluye como "autorizadas" a las estaciones con autorización vigente o que se encuentran en proceso de renovación, y no cuentan con resolución que la deje sin efecto o extinción firme. El estado “cautelar” corresponde a estaciones cuya autorización ha sido cancelada en la vía administrativa, pero cuentan con una medida cautelar otorgada por el Poder Judicia.

\subsection{Número de estaciones de Transmisión FM}

\begin{tabular}{|l|l|} \hline
	Canalización 			& 28 \\ \hline
	Autorizaciones			& 25 \\ \hline
	Cautelar				& 0 \\ \hline
	Reservado Estado		& 1 \\ \hline
	Disponibilidad			& 2 \\ \hline
	Trámite					& 1 \\ \hline 
\end{tabular}

\subsection{Límite de potencia de transmisión}

La Potencia Efectiva Radiada máxima en la dirección de máxima ganancia de antena a ser autorizada en esta localidad será: 1 KW.

\subsection{Límites geográficos de autorización}

La transmisioón está autorizada para llegar hasta la provincia de Canchis, el distrito/centro poblado:

\begin{itemize}
	\item Combapata
	\item Marangani
	\item San Pablo
	\item San Pedro
	\item Sicuani
	\item Tinta
\end{itemize}



\section{Chumbivilcas - Santo Tomas}

\begin{itemize}
	\item Plan Aprobado por RVM Nº 426-2014-MTC/03 (28/07/2014)
	\item Con Restricción RD Nº 1676-2017-MTC/28 (18/08/2017)
\end{itemize}

El Registro Nacional de Frecuencias incluye como "autorizadas" a las estaciones con autorización vigente o que se encuentran en proceso de renovación, y no cuentan con resolución que la deje sin efecto o extinción firme. El estado “cautelar” corresponde a estaciones cuya autorización ha sido cancelada en la vía administrativa, pero cuentan con una medida cautelar otorgada por el Poder Judicia.

\subsection{Número de estaciones de Transmisión FM}

\begin{tabular}{|l|l|} \hline
	Canalización 			& 20 \\ \hline
	Autorizaciones			& 17 \\ \hline
	Cautelar				& 0 \\ \hline
	Reservado Estado		& 1 \\ \hline
	Disponibilidad			& 2 \\ \hline
	Trámite					& 0 \\ \hline 
\end{tabular}

\subsection{Límite de potencia de transmisión}

La Potencia Efectiva Radiada máxima en la dirección de máxima ganancia de antena a ser autorizada en esta localidad será: 0.25 KW.

\subsection{Límites geográficos de autorización}

La transmisioón está autorizada para llegar hasta la provincia de Chumbivilcas, el distrito/centro poblado:

\begin{itemize}
	\item Santo Tomas
\end{itemize}



\section{Cusco - Cusco}

\begin{itemize}
	\item Plan Aprobado por RVM Nº 108-2004-MTC/03 (10/07/2004)
	\item Con Restricción RD Nº 1164-2006-MTC/17 (28/06/2006)
	\item Concurso Público Nº 01-2015-MTC/28 (30/12/2015), RD Nº 2129-2015-MTC/28 (29/12/2015)
\end{itemize}

El Registro Nacional de Frecuencias incluye como "autorizadas" a las estaciones con autorización vigente o que se encuentran en proceso de renovación, y no cuentan con resolución que la deje sin efecto o extinción firme. El estado “cautelar” corresponde a estaciones cuya autorización ha sido cancelada en la vía administrativa, pero cuentan con una medida cautelar otorgada por el Poder Judicia.

\subsection{Número de estaciones de Transmisión FM}

\begin{tabular}{|l|l|} \hline
	Canalización 			& 28 \\ \hline
	Autorizaciones			& 24 \\ \hline
	Cautelar				& 0 \\ \hline
	Reservado Estado		& 1 \\ \hline
	Disponibilidad			& 3 \\ \hline
	Trámite					& 1 \\ \hline 
\end{tabular}

\subsection{Límite de potencia de transmisión}

La Potencia Efectiva Radiada máxima en la dirección de máxima ganancia de antena a ser autorizada en esta localidad será: 1 KW.

\subsection{Límites geográficos de autorización}

La transmisioón está autorizada para llegar hasta la provincia de Anta y Cusco, el distrito/centro poblado:

\begin{itemize}
	\item Cachimayo
	\item Cusco
	\item Poroy
	\item Poroy
	\item San Jeronimo
	\item San Sebastian
	\item Santiago
	\item Saylla
	\item Wanchaq
\end{itemize}



\section{Espinar - Espinar}

\begin{itemize}
	\item Plan Aprobado por RVM Nº 426-2014-MTC/03 (28/07/2014)
	\item Con Restricción RD Nº 1135-2012-MTC/28 (26/09/2012)
	\item Concurso Público Nº CP 02-2013-MTC/28 (18/12/2013), RD Nº 1841-2013-MTC/28 (12/12/2013)
\end{itemize}

El Registro Nacional de Frecuencias incluye como "autorizadas" a las estaciones con autorización vigente o que se encuentran en proceso de renovación, y no cuentan con resolución que la deje sin efecto o extinción firme. El estado “cautelar” corresponde a estaciones cuya autorización ha sido cancelada en la vía administrativa, pero cuentan con una medida cautelar otorgada por el Poder Judicia.

\subsection{Número de estaciones de Transmisión FM}

\begin{tabular}{|l|l|} \hline
	Canalización 			& 29 \\ \hline
	Autorizaciones			& 22 \\ \hline
	Cautelar				& 0 \\ \hline
	Reservado Estado		& 2 \\ \hline
	Disponibilidad			& 5 \\ \hline
	Trámite					& 0 \\ \hline 
\end{tabular}

\subsection{Límite de potencia de transmisión}

La Potencia Efectiva Radiada máxima en la dirección de máxima ganancia de antena a ser autorizada en esta localidad será: 1 KW.

\subsection{Límites geográficos de autorización}

La transmisioón está autorizada para llegar hasta la provincia de Espinar, el distrito/centro poblado:

\begin{itemize}
	\item Alto Pichigua
	\item Coporaque
	\item Espinar
	\item Pichigua
\end{itemize}



\section{La convención - Santa Ana}

\begin{itemize}
	\item Plan Aprobado por RVM Nº 108-2004-MTC/03 (10/07/2004)
	\item Con Restricción RD Nº 3548-2010-MTC/28 (12/11/2010)
	\item Concurso Público Nº 02-2019-MTC/28 (23/09/2019), RD Nº 2770-2019-MTC/28 (20/09/2019)
\end{itemize}

El Registro Nacional de Frecuencias incluye como "autorizadas" a las estaciones con autorización vigente o que se encuentran en proceso de renovación, y no cuentan con resolución que la deje sin efecto o extinción firme. El estado “cautelar” corresponde a estaciones cuya autorización ha sido cancelada en la vía administrativa, pero cuentan con una medida cautelar otorgada por el Poder Judicia.

\subsection{Número de estaciones de Transmisión FM}

\begin{tabular}{|l|l|} \hline
	Canalización 			& 28 \\ \hline
	Autorizaciones			& 19 \\ \hline
	Cautelar				& 0 \\ \hline
	Reservado Estado		& 1 \\ \hline
	Disponibilidad			& 8 \\ \hline
	Trámite					& 1 \\ \hline 
\end{tabular}

\subsection{Límite de potencia de transmisión}

La Potencia Efectiva Radiada máxima en la dirección de máxima ganancia de antena a ser autorizada en esta localidad será: 1 KW.

\subsection{Límites geográficos de autorización}

La transmisioón está autorizada para llegar hasta la provincia de La Convención, el distrito/centro poblado:

\begin{itemize}
	\item Maranura
	\item Santa Ana
\end{itemize}



\section{Paruro - Paruro}

\begin{itemize}
	\item Plan Aprobado por RVM Nº 426-2014-MTC/03 (28/07/2014)
	\item Con Restricción RD Nº 0265-2017-MTC/28 (15/03/2017)
\end{itemize}

El Registro Nacional de Frecuencias incluye como "autorizadas" a las estaciones con autorización vigente o que se encuentran en proceso de renovación, y no cuentan con resolución que la deje sin efecto o extinción firme. El estado “cautelar” corresponde a estaciones cuya autorización ha sido cancelada en la vía administrativa, pero cuentan con una medida cautelar otorgada por el Poder Judicia.

\subsection{Número de estaciones de Transmisión FM}

\begin{tabular}{|l|l|} \hline
	Canalización 			& 16 \\ \hline
	Autorizaciones			& 8 \\ \hline
	Cautelar				& 0 \\ \hline
	Reservado Estado		& 2 \\ \hline
	Disponibilidad			& 6 \\ \hline
	Trámite					& 0 \\ \hline 
\end{tabular}

\subsection{Límite de potencia de transmisión}

La Potencia Efectiva Radiada máxima en la dirección de máxima ganancia de antena a ser autorizada en esta localidad será: 0.5 KW.

\subsection{Límites geográficos de autorización}

La transmisioón está autorizada para llegar hasta la provincia de Paruro, el distrito/centro poblado:

\begin{itemize}
	\item Rondocan
	\item Colcha
	\item Paruro
\end{itemize}



\section{Paucartambo - Paucartambo}

\begin{itemize}
	\item Plan Aprobado por RVM Nº 426-2014-MTC/03 (28/07/2014)
	\item Con Restricción RD Nº 1953-2016-MTC/28 (17/12/2016)
\end{itemize}

El Registro Nacional de Frecuencias incluye como "autorizadas" a las estaciones con autorización vigente o que se encuentran en proceso de renovación, y no cuentan con resolución que la deje sin efecto o extinción firme. El estado “cautelar” corresponde a estaciones cuya autorización ha sido cancelada en la vía administrativa, pero cuentan con una medida cautelar otorgada por el Poder Judicia.

\subsection{Número de estaciones de Transmisión FM}

\begin{tabular}{|l|l|} \hline
	Canalización 			& 15 \\ \hline
	Autorizaciones			& 8 \\ \hline
	Cautelar				& 0 \\ \hline
	Reservado Estado		& 0 \\ \hline
	Disponibilidad			& 7 \\ \hline
	Trámite					& 0 \\ \hline 
\end{tabular}

\subsection{Límite de potencia de transmisión}

La Potencia Efectiva Radiada máxima en la dirección de máxima ganancia de antena a ser autorizada en esta localidad será: 0.5 KW.

\subsection{Límites geográficos de autorización}

La transmisioón está autorizada para llegar hasta la provincia de Paucartambo, el distrito/centro poblado:

\begin{itemize}
	\item Challabamba
	\item Colquepata
	\item Paucartambo
\end{itemize}



\section{Quispicanchi - Urcos}

\begin{itemize}
	\item Plan Aprobado por RVM Nº 1141-2016-MTC/03 (27/07/2016)
	\item Con Restricción RD Nº 1776-2012-MTC/28 (18/01/2013)
	\item Concurso Público Nº 01-2015-MTC/28 (30/12/2015), RD Nº 2129-2015-MTC/28 (29/12/2015)
\end{itemize}

El Registro Nacional de Frecuencias incluye como "autorizadas" a las estaciones con autorización vigente o que se encuentran en proceso de renovación, y no cuentan con resolución que la deje sin efecto o extinción firme. El estado “cautelar” corresponde a estaciones cuya autorización ha sido cancelada en la vía administrativa, pero cuentan con una medida cautelar otorgada por el Poder Judicia.

\subsection{Número de estaciones de Transmisión FM}

\begin{tabular}{|l|l|} \hline
	Canalización 			& 26 \\ \hline
	Autorizaciones			& 21 \\ \hline
	Cautelar				& 0 \\ \hline
	Reservado Estado		& 2 \\ \hline
	Disponibilidad			& 3 \\ \hline
	Trámite					& 0 \\ \hline 
\end{tabular}

\subsection{Límite de potencia de transmisión}

La Potencia Efectiva Radiada máxima en la dirección de máxima ganancia de antena a ser autorizada en esta localidad será: 0.5 KW.

\subsection{Límites geográficos de autorización}

La transmisioón está autorizada para llegar hasta la provincia de Paucartambo y Quispicanchi, el distrito/centro poblado:

\begin{itemize}
	\item Caicay
	\item Andahuaylillas
	\item Huaro
	\item Lucre
	\item Oropesa
	\item Urcos
\end{itemize}



\section{Urubamba - Urubamba}

\begin{itemize}
	\item Plan Aprobado por RVM Nº 108-2004-MTC/03 (10/07/2004)
	\item Con Restricción RD Nº 2189-2005-MTC/17 (12/01/2006)
	\item Concurso Público Nº 02-2012-MTC/28 (17/11/2012), RD Nº 1555-2012-MTC/28 (07/11/2012)
\end{itemize}

El Registro Nacional de Frecuencias incluye como "autorizadas" a las estaciones con autorización vigente o que se encuentran en proceso de renovación, y no cuentan con resolución que la deje sin efecto o extinción firme. El estado “cautelar” corresponde a estaciones cuya autorización ha sido cancelada en la vía administrativa, pero cuentan con una medida cautelar otorgada por el Poder Judicia.

\subsection{Número de estaciones de Transmisión FM}

\begin{tabular}{|l|l|} \hline
	Canalización 			& 28 \\ \hline
	Autorizaciones			& 24 \\ \hline
	Cautelar				& 0 \\ \hline
	Reservado Estado		& 2 \\ \hline
	Disponibilidad			& 2 \\ \hline
	Trámite					& 1 \\ \hline 
\end{tabular}

\subsection{Límite de potencia de transmisión}

La Potencia Efectiva Radiada máxima en la dirección de máxima ganancia de antena a ser autorizada en esta localidad será: 0.5 KW.

\subsection{Límites geográficos de autorización}

La transmisioón está autorizada para llegar hasta la provincia de Anta y Urubamba, el distrito/centro poblado:

\begin{itemize}
	\item Arcahuasi
	\item Anta
	\item Huarocondo
	\item Pucyura
	\item Zurite
	\item Chinchero
	\item Huayllabamba
	\item Maras
	\item Ollantaytambo
	\item Urubamba
	\item Yucay
\end{itemize}


\section{Procedimiento para obtener autorización de emisión en radiodifusión}

\subsection{Conceptos Generales}
Es un servicio de telecomunicaciones, cuya 
programación es recibida por el público en 
general. 

Se clasifica en 

\begin{itemize}
	\item Por la modalidad de operación
	\begin{itemize}
		\item Sonora: a su vez se clasifica en: 
		Frecuencia Modulada, Onda Media y 
		Onda Corta (Tropical e Internacional).
		\item Por televisión: a  su  vez  se  clasifica  en: 
		VHF y UHF
	\end{itemize}
	\item Por la finalidad:
	\begin{itemize}
		\item Comercial
		\item Educativa
		\item Comunitaria
	\end{itemize}
\end{itemize}

\subsubsection{Radiodifusión Comercial}
La Radiodifusión es aquella cuyos programas contienen temas 
relacionados  con  el  entretenimiento,  recreación, 
información,  noticiosos  y  de  orientación  de  la 
comunidad.

\subsubsection{Radiodifusión Comercial}
Es aquella cuyos programas contienen temas 
relacionados,  principalmente,  con  la  educación, 
cultura y deporte. 

\subsubsection{Radiodifusión Educativa}

Es aquella cuyos programas contienen temas 
relacionados, principalmente, con el fomento de 
la identidad, costumbres de la comunidad e 
integración nacional. La estación radiodifusora  debe necesariamente 
ubicarse  en  comunidades  campesinas,  nativas  e 
indígenas,  áreas  rurales  o  de  preferente  interés 
social. 

\subsection{Procedimiento}

Cualquier persona sea natural o persona jurídica puede prestar el servicio . Para la prestación del servicio de radiodifusión se 
requiere  contar  con  una  autorización  otorgada 
por resolución del Viceministro de 
Comunicaciones.

Para la prestación de servicio se necesita una autorización que es la que faculta la prestación de un servicio de radiodifusión. La solicitud se presenta 
ante  la  Dirección  General  de  Autorizaciones  en 
Telecomunicaciones.  Para  tal  efecto  se  deberá 
emplear los formatos que están disponibles en la 
página web del MTC (www.mtc.gob.pe)

Las autorizaciones se otorgan por un plazo 
máximo de diez (10) años. 

\subsubsection{Procedimientos están sujetos al silencio  administrativo positivo}

\begin{itemize}
	\item Transferencia de autorización. 
	\item Renovación de autorización. 
	\item Transferencia de acciones, participaciones, 
	titularidad y modificación del representante 
	legal, directorio o consejo directivo.
\end{itemize}

\subsubsection{Procedimientos están sujetos al silencio administrativo negativo}
\begin{itemize}
	\item Autorización para prestar el servicio de 
	radiodifusión.
	\item Modificación de características técnicas 
	(aumento de potencia y cambio de ubicación de 
	la  planta  transmisora)  y  de  condición  esencial 
	(frecuencia o canal asignado). 
	\item Otorgamiento de pago fraccionado.
	\item Otorgamiento de nuevo pago fraccionado.
\end{itemize}

\subsubsection{Las modalidades de otorgamiento de una Autorización}
\begin{enumerate}
	\item A solicitud de Parte
	\item Por Concurso público
\end{enumerate}

El otorgamiento de 
autorización a solicitud de parte procede cuando el número de frecuencias o canales 
disponibles  en  una  misma  banda  y  localidad  es 
superior a las solicitudes admitidas.

Una Autorización por concurso público se otorga cuando existe restricción en la disponibilidad de 
frecuencias  o  canales  en  una  misma  banda  y 
localidad. 

La restricción en la disponibilidad de frecuencias o canales es  cuando  el  número  de  frecuencias  o  canales 
disponibles  es  inferior  al  número  de  solicitudes 
admitidas.

Las frecuencias o 
canales que pueden autorizarse en una 
determinada localidad se  enumeran  en  los  Planes  de  Canalización  y 
Asignación de Frecuencias, allí se indican el 
máximo de frecuencias o canales que pueden ser 
asignados en una localidad. 

La  situación  de  restricción  se  configura  en  la 
fecha  de  ingreso  de  la  solicitud  que  supera  el 
número de frecuencias o canales disponibles.  

Una vez declarada una restricción en una localidad y banda, Dentro de los 15 días siguientes a la declaratoria, la Dirección General de 
Autorizaciones en Telecomunicaciones debe 
denegar todas las solicitudes que se encontraban en trámite, con independencia del estado  de las 
mismas.

\subsubsection{Causales para denegar una autorización}

\begin{itemize}
	\item  Tener  más  del  20 \%  de  frecuencias  asignadas 
	en el servicio de radiodifusión sonora. 
	\item Tener más del 30\% de canales asignados en el 
	servicio de radiodifusión por televisión.
	\item Tener  adeudos  por  tasa,  canon,  derechos  o 
	multas. 
	\item Haber sido condenado con pena 
	privativa  de  la  libertad  mayor  de  4  años  por 
	delito  doloso. 
	\item Estar  inhabilitado  para  contratar 
	con el Estado con resolución firme. 
	\item Haber  sido  sancionado  más  de  tres  veces  por 
	infracciones muy graves, en el lapso de diez años 
	anteriores a la fecha de la solicitud, por 
	resolución  con  autoridad  de  cosa  decidida  en  la 
	misma localidad.
\end{itemize}

\subsubsection{Prohibiciones para acceder a una autorización}

\begin{itemize}
	\item Ejercer cargos gubernamentales, sea en el 
	Poder  Ejecutivo,  Legislativo  o  Judicial,  así  como 
	en organismos reguladores, organismos públicos 
	descentralizados, directivos de empresas del 
	Estado, Alcaldes a título personal.
	\item Funcionarios  vinculados  a  la  gestión  y  control 
	del servicio de radiodifusión. 
	\item El cónyuge, conviviente integrante de la unión 
	de hecho, y los parientes hasta el segundo grado de consanguinidad, de las personas a que se 
	refieren los numerales precedentes. 
\end{itemize}

\subsubsection{Requisitos que debe acompañarse a una solicitud de otorgamiento de autorización}

Son de 4 tipos: Legales, técnicos, económicos y de comunicación. Los requisitos están 
contenidos  en  el  Texto  Único  de  Procedimientos 
Administrativos – TUPA del MTC. 

\subsubsection{Requisitos Legales}

\begin{itemize}
	\item Para el caso de personas naturales:
	\begin{itemize}
		\item Número de Documento Nacional de Identidad. 
		\item Hoja de Datos Personales. 
		\item Declaración jurada de no haber sido 
		condenado a pena privativa de libertad de cuatro 
		(4) años o más, por delito doloso. 
	\end{itemize}
	\item Para el caso de personas jurídicas: 
	\begin{itemize}
		\item Número del documento de identidad del 
		representante legal. 
		\item Declaración  Jurada  de  vigencia  del  poder  del 
		representante legal. 
		\item Copia  del  instrumento  legal  donde  conste  la 
		calidad  de  socio,  accionista  o  asociado,  según 
		sea el caso. 
		\item Declaración  Jurada  de  Relación  de  Miembros. 
		(Indicar  la  composición  societaria  o  accionaria  o 
		la relación de asociados).
		\item Hoja de datos personales de socios, accionistas, 
		asociados,  titular,  representante  legal,  gerente, 
		apoderado  y  directores  nacionales  y  extranjeros, 
		de ser el caso. 
		\item Declaración jurada de no haber sido 
		condenado a pena privativa de libertad de cuatro 
		(4) años o más, por delito doloso.
	\end{itemize}
\end{itemize}

\subsubsection{Requisitos Técnicos}

\begin{itemize}
	\item Perfil  del  proyecto  técnico  de  la  estación  a 
	instalar,  autorizado  por  un  ingeniero  colegiado 
	de la especialidad, habilitado a la fecha de 
	presentación de la solicitud. 
	\item Plano a escala 1/100,000 de la localidad en la 
	que va a operar la estación, graficando la zona 
	de servicio. Este requisito no es exigible, 
	tratándose de solicitudes para prestar el servicio 
	de radiodifusión en la banda de onda corta.
\end{itemize}

\subsubsection{Requisitos Económicos}

Inversión proyectada del primer año para la 
instalación de la estación radiodifusora. 

\subsubsection{Requisito de Comunicación}
El proyecto de comunicación. 

\subsubsection{Procedimiento de Otorgamiento de una Autorización}
\begin{itemize}
	\item En Mesa de Partes del MTC:  
	\begin{itemize}
		\item La  solicitud  se  presenta  con  todos  los 
		requisitos establecidos en el TUPA. 
		\item Si  se  adjuntan  todos  los  requisitos,  la 
		solicitud se deriva a la Dirección General 
		de Autorizaciones en 
		Telecomunicaciones. 
		\item Si falta algún requisito, se le comunica al 
		usuario  el  documento  que  falta  dando 
		48  horas  de  plazo  para  presentarlo.  En 
		la solicitud y su copia se indica el 
		documento que falta y el plazo 
		otorgado.
		\item Mientras  esté  pendiente  de  subsanar  la 
		solicitud,  no  se  remite  la  solicitud  a  la 
		DGAT.
		\item  Si  vencido  el  plazo,  no  se  presenta  el 
		documento subsanado, se considera 
		como  no  presentada  o  no  admitida  la 
		respectiva solicitud.
	\end{itemize}
	\item En  la  Dirección  General  de  Autorizaciones  en 
	Telecomunicaciones:
	\begin{itemize}
		\item Se verificación la disponibilidad de 
		frecuencias  en  la  localidad  y  banda  de 
		frecuencias solicitada.
		\item De  no  existir  restricción  se  continúa  el 
		trámite.
		\item  De  configurarse  restricción,  se  deniega 
		la solicitud. 
		\item De ser necesario se puede requerir 
		información adicional al solicitante.
		\item Solicita información al órgano 
		competente  respecto  de  las  sanciones  y 
		los pagos. 
		\item Concluida la evaluación:
		\begin{itemize}
			\item De ser improcedente, se elabora un informe 
			(técnico legal) y proyecto de Resolución 
			Directoral denegando la solicitud. 
			\item De resultar procedente, se elabora un 
			informe (técnico-legal) y proyecto de 
			Resolución  Viceministerial,  el  que  se  remite 
			a  la  Oficina  General  de  Asesoría  Jurídica, 
			para  su  revisión  y  de  estar  conforme  lo 
			deriva  al  Viceministro  de  Comunicaciones 
			para la firma de la resolución. 
		\end{itemize}
	\end{itemize}
\end{itemize}

\subsubsection{Obligaciones que debe cumplir el titular de una autorización}
Pagar:
\begin{itemize}
	\item El  derecho  de  autorización  y  canon  dentro  de 
	los  60  días  posteriores  a  la  notificación  de  la 
	Resolución Viceministerial. 
	\item El canon por la asignación del espectro 
	radioeléctrico, en el mes de febrero. 
	\item La tasa por la explotación comercial de 
	servicio, en forma mensual y efectuar la 
	liquidación  final  en  el  mes  de  abril.  (En  caso  de 
	tener finalidad comercial)
\end{itemize}

\subsubsection{Periodo de instalación y prueba}
Es  el  plazo  dentro  del  cual,  el  titular  instala  los 
equipos requeridos para la prestación del servicio 
autorizado y se realizan las pruebas de 
funcionamiento respectivas. El período de instalación y prueba dura doce (12) 
meses  contados  a  partir  del  día  siguiente  de  la 
fecha de notificación de la resolución de 
autorización,  prorrogables  por  el  plazo  de  seis 
meses, a solicitud del titular. 

\subsubsection{Prohibiciones para la instalación de plantas de transmisión}

Las  plantas  transmisoras  de  las  estaciones  de 
radiodifusión no podrán instalarse en:

\begin{enumerate}
	\item En  las  zonas  de  restricción  determinadas  por 
	la  Dirección  General  de  Aeronáutica  Civil,  de 
	conformidad con las normas de la materia. 
	\item En las zonas de restricción cercana a las 
	Estaciones de Comprobación Técnica que 
	forman  parte  del  Sistema  Nacional  de  Gestión  y 
	Control  del  Espectro  Radioeléctrico,  de  acuerdo 
	a las normas correspondientes. 
	\item En las proximidades de las plantas de 
	fabricación o almacenamiento de explosivos o de 
	abastecimiento de combustible derivados del 
	petróleo  y  gas,  de  acuerdo  a  las  normas  que 
	apruebe al Ministerio. 
	\item En las zonas declaradas como patrimonio 
	cultural por el Instituto Nacional de Cultura. 
	\item En  las  áreas  naturales  protegidas  declaradas 
	por el Instituto Nacional de Recursos Naturales. 
	\item En las zonas cercanas a los observatorios 
	geofísicos y radio observatorios. 
	\item En otras zonas que determine el Ministerio. En  los  supuestos  señalados  en  los  numerales  1), 
	4), 5) y 6) del presente artículo, podrá otorgarse 
	autorización  si  el  solicitante  cuenta  previamente 
	con la conformidad de la autoridad competente. 
\end{enumerate}

\subsubsection{Causales para dejar sin efecto una autorización}

La autorización quedará sin efecto por:
\begin{enumerate}
	\item  Incumplimiento  de  las  obligaciones  derivadas 
	del período de instalación y prueba. 
	\item Incumplimiento,  por más  de  dos  (2)  años 
	consecutivos, del pago del canon por la 
	utilización  del  espectro  radioeléctrico  o  la  tasa 
	por explotación comercial del servicio.  
	\item Suspensión de la prestación del servicio 
	(operaciones) por más de 3 meses consecutivos o 
	5 alternados en el lapso de 1 año.  
	\item Renuncia del titular de la autorización. 
	\item Por incumplimiento de lo dispuesto en el 
	artículo 20 de la Ley, previo requerimiento. 
	\item Incumplimiento de las obligaciones 
	establecidas en las bases del concurso público. 
\end{enumerate}

\subsubsection{Causales de extinción de una autorización}
\begin{enumerate}
	\item Muerte,  extinción  o  declaratoria  de  quiebra 
	del titular, según sea el caso.
	\item El  vencimiento  del  plazo  de  vigencia,  salvo 
	que se haya solicitado la respectiva renovación o 
	se verifique la continuidad  de la operación de la 
	estación radiodifusora, conforme lo establezca el 
	Reglamento. 
\end{enumerate}



\end{document}

