\documentclass[11pt]{article}
\usepackage[spanish]{babel}
\usepackage[utf8]{inputenc}
\usepackage{amsmath}
\usepackage{listings}
\usepackage[usenames]{color} %seteamos el uso de nombre y color
\definecolor{gray97}{gray}{.97}%definimos nombre y color
\usepackage{textcomp}
\lstset{
	frame=Ltb,
	framerule=1pt,
	framextopmargin=5pt, %margen de arriba
	framexbottommargin=5pt, %margen de abajo
	framexleftmargin= 2pt, %separacion del margen izquierdo
	framesep=5pt,
	rulesep=0.3pt,
	backgroundcolor=\color{gray97},
	rulesepcolor=,
	tabsize=2,
	rulecolor=\color[RGB]{106, 182, 217}, %AZUL
	upquote=true,
	aboveskip={2\baselineskip}, %despues de la linea de texto
	columns=fixed,
	showstringspaces=false,
	extendedchars=true,
	breaklines=true,
	prebreak = \raisebox{0ex}[0ex][0ex]{\ensuremath{\hookleftarrow}},
	showtabs=false,
	showspaces=false,
	showstringspaces=false,
	basicstyle=\scriptsize\ttfamily\color[RGB]{39, 100, 46}, %Numeros de lineas, simbolos, puntos y coma y demas
	identifierstyle=\ttfamily\color[RGB]{56, 140, 189}, %variables
	commentstyle=\color[RGB]{62, 179, 101}, %comentarios
	stringstyle=\color[RGB]{247, 165, 42}, %impresiones
	keywordstyle=\bfseries\color[RGB]{237, 118, 150}, %funciones
	%
	numbers=left,
	numbersep=1pt, %separacion del numero
	numberstyle=\tiny,
	numberfirstline = false,
	breaklines=true,
}
\usepackage{textcomp}
\usepackage{graphicx}
\usepackage[colorinlistoftodos]{todonotes}
\usepackage{eso-pic}
\usepackage{avant}
\usepackage[top=2cm,bottom=2cm,left=2.5cm,right=2.5cm,headsep=8pt,a4paper]{geometry}
\usepackage{fancyhdr}
\pagestyle{fancy}
\fancyhf{}
%\fancyhead[LE,RO]{UNSAAC 2019-II}
%\fancyhead[RE,LO]{Ingenieria Electrónica}
%\fancyfoot[CE,CO]{\leftmark}
\fancyfoot[CO]{\thepage}
\renewcommand{\headrulewidth}{1pt}
\renewcommand{\footrulewidth}{1pt}
\usepackage{tabu}
\usepackage{array}
\usepackage{multirow}
\usepackage{amssymb}
\usepackage{makeidx}
\usepackage{wrapfig}
\usepackage{enumerate}
\usepackage{amsmath,tikz}
\usepackage{steinmetz}
\newcommand*{\horzbar}{\rule[0.05ex]{2.5ex}{0.5pt}}
\usepackage{calc}
\usepackage{dsfont}
\usepackage{enumitem}
\usepackage{subfig}
\usepackage{tabularx}

\title{
	\textsc{Universidad Nacional de San Antonio Abad del Cusco}\\
	\textbf{Compuertas Lógicas}\\
	Preinforme 1}

\author{
	\begin{tabular}{lr}
		Edison \textsc{Abado Ancco} & 145012 \\
	\end{tabular}
}


\begin{document}
	
	\begin{titlepage}
		\newcommand{\HRule}{\rule{\linewidth}{0.5mm}} 
		\center
		\textsc{\LARGE  Universidad Nacional de San \\[0.2cm] Antonio Abad del Cusco}\\[1.5cm] 
		\includegraphics[width=4cm]{IMAGENES/escudo}\\[1cm]
		\textsc{\Large Facultad de Ingeniería Eléctrica, \\ Electrónica, Informática y Mecánica}\\[0.5cm] 
		\textsc{\large Escuela Profesional de Ingeniería Electrónica}\\[0.5cm]
		\textsc{\Large \textbf{Telecomunicaciones I}}\\[0.5cm] 
		\HRule \\[0.4cm]
		{ \huge \bfseries Radiodifusión sonora FM en la región de Cusco}\\[0.4cm] 
		\HRule \\[1.5cm]
		\begin{minipage}{\textwidth}
			\center 
			
			\emph{Profesor:} \\
			Ing. Fernando \textsc{Tagle Carbajal } \\[1cm]
			
			\begin{tabular}{ll}
				\emph{Alumno:} & \emph{Código:}\\
				Edison \textsc{Abado Ancco} & 145012 \\
			\end{tabular}
		\end{minipage}\\[2cm]
		\today
	\end{titlepage}
	
	
	\newpage
	
	
\section{Acomayo - Acomayo}

\begin{itemize}
	\item Plan Aprobado por RVM Nº 426-2014-MTC/03 (28/07/2014)
	\item Con Restricción RD Nº 1676-2017-MTC/28 (18/08/2017)
\end{itemize}

El Registro Nacional de Frecuencias incluye como "autorizadas" a las estaciones con autorización vigente o que se encuentran en proceso de renovación, y no cuentan con resolución que la deje sin efecto o extinción firme. El estado “cautelar” corresponde a estaciones cuya autorización ha sido cancelada en la vía administrativa, pero cuentan con una medida cautelar otorgada por el Poder Judicia.

\subsection{Número de estaciones de Transmisión FM}

\begin{tabular}{|r|r|} \hline
	Canalización 			& 20 \\ \hline
	Autorizaciones			& 16 \\ \hline
	Cautelar				& 0 \\ \hline
	Reservado Estado		& 1 \\ \hline
	Disponibilidad			& 3 \\ \hline
	Trámite					& 2 \\ \hline 
\end{tabular}

\subsection{Límite de potencia de transmisión}

La Potencia Efectiva Radiada máxima en la dirección de máxima ganancia de antena a ser autorizada en esta localidad será: 0.5 KW.
	
\subsection{Límites geográficos de autorización}

La transmisioón está autorizada Para La provincia de Acomayo, y los distritos:

\begin{itemize}
	\item Acomayo 
	\item Acopia 
	\item Pomacanchi 
	\item Sangarará
\end{itemize}
	


\section{Anta - Anta}

\begin{itemize}
	\item Plan Aprobado por RVM Nº 108-2004-MTC/03 (10/07/2004)
	\item Con Restricción RD Nº 2189-2005-MTC/17 (12/01/2006)
	\item Concurso Público Nº 02-2012-MTC/28 (17/11/2012), RD Nº 1555-2012-MTC/28 (07/11/2012)
\end{itemize}

El Registro Nacional de Frecuencias incluye como "autorizadas" a las estaciones con autorización vigente o que se encuentran en proceso de renovación, y no cuentan con resolución que la deje sin efecto o extinción firme. El estado “cautelar” corresponde a estaciones cuya autorización ha sido cancelada en la vía administrativa, pero cuentan con una medida cautelar otorgada por el Poder Judicia.

\subsection{Número de estaciones de Transmisión FM}

\begin{tabular}{|r|r|} \hline
	Canalización 			& 20 \\ \hline
	Autorizaciones			& 24 \\ \hline
	Cautelar				& 0 \\ \hline
	Reservado Estado		& 2 \\ \hline
	Disponibilidad			& 2 \\ \hline
	Trámite					& 1 \\ \hline 
\end{tabular}

\subsection{Límite de potencia de transmisión}

La Potencia Efectiva Radiada máxima en la dirección de máxima ganancia de antena a ser autorizada en esta localidad será: 0.5 KW.

\subsection{Límites geográficos de autorización}

La transmisioón está autorizada para llegar hasta la provincia de Anta y Urubamba, los distritos/centro poblado:

\begin{itemize}
	\item Ancahuasi
	\item Anta 
	\item Huarocondo 
	\item Pucyura
	\item Zurite
	\item Chinchero
	\item Huayllabamba
	\item Maras
	\item Ollantaytambo
	\item Urubamba
	\item Yucay
\end{itemize}




\section{Calca - Calca}

\begin{itemize}
	\item Plan Aprobado por RVM Nº 108-2004-MTC/03 (10/07/2004)
	\item Con Restricción RD Nº 067-2011-MTC/28 (29/01/2011)
	\item Concurso Público Nº CP 02-2013-MTC/28 (18/12/2013), RD Nº 1841-2013-MTC/28 (12/12/2013)
\end{itemize}

El Registro Nacional de Frecuencias incluye como "autorizadas" a las estaciones con autorización vigente o que se encuentran en proceso de renovación, y no cuentan con resolución que la deje sin efecto o extinción firme. El estado “cautelar” corresponde a estaciones cuya autorización ha sido cancelada en la vía administrativa, pero cuentan con una medida cautelar otorgada por el Poder Judicia.

\subsection{Número de estaciones de Transmisión FM}

\begin{tabular}{|r|r|} \hline
	Canalización 			& 14 \\ \hline
	Autorizaciones			& 12 \\ \hline
	Cautelar				& 0 \\ \hline
	Reservado Estado		& 1 \\ \hline
	Disponibilidad			& 1 \\ \hline
	Trámite					& 1 \\ \hline 
\end{tabular}

\subsection{Límite de potencia de transmisión}

La Potencia Efectiva Radiada máxima en la dirección de máxima ganancia de antena a ser autorizada en esta localidad será: 0.5 KW.

\subsection{Límites geográficos de autorización}

La transmisioón está autorizada para llegar hasta la provincia de Calca, los distritos/centro poblado:

\begin{itemize}
	\item Calca
	\item Calca - Totora
	\item Coya
	\item Lamay
\end{itemize}


\section{Canas - Yanaoca}

\begin{itemize}
	\item Plan Aprobado por RVM Nº 108-2004-MTC/03 (10/07/2004)
	\item Con Restricción RD Nº 2908-2017-MTC/28 (19/01/2018)
\end{itemize}

El Registro Nacional de Frecuencias incluye como "autorizadas" a las estaciones con autorización vigente o que se encuentran en proceso de renovación, y no cuentan con resolución que la deje sin efecto o extinción firme. El estado “cautelar” corresponde a estaciones cuya autorización ha sido cancelada en la vía administrativa, pero cuentan con una medida cautelar otorgada por el Poder Judicia.

\subsection{Número de estaciones de Transmisión FM}

\begin{tabular}{|r|r|} \hline
	Canalización 			& 15 \\ \hline
	Autorizaciones			& 10 \\ \hline
	Cautelar				& 0 \\ \hline
	Reservado Estado		& 1 \\ \hline
	Disponibilidad			& 4 \\ \hline
	Trámite					& 1 \\ \hline 
\end{tabular}

\subsection{Límite de potencia de transmisión}

La Potencia Efectiva Radiada máxima en la dirección de máxima ganancia de antena a ser autorizada en esta localidad será: 0.5 KW.

\subsection{Límites geográficos de autorización}

La transmisioón está autorizada para llegar hasta la provincia de Canas, el distrito/centro poblado:

\begin{itemize}
	\item Yanaoca
\end{itemize}



\section{Canchis - Sicuani}

\begin{itemize}
	\item Plan Aprobado por RVM Nº 108-2004-MTC/03 (10/07/2004)
	\item Con Restricción RD Nº 1806-2006-MTC/17 (02/09/2006)
	\item Concurso Público Nº 01-2020-MTC/28 (30/12/2020), RD Nº 1891-2020-MTC/28 (30/12/2020)
\end{itemize}

El Registro Nacional de Frecuencias incluye como "autorizadas" a las estaciones con autorización vigente o que se encuentran en proceso de renovación, y no cuentan con resolución que la deje sin efecto o extinción firme. El estado “cautelar” corresponde a estaciones cuya autorización ha sido cancelada en la vía administrativa, pero cuentan con una medida cautelar otorgada por el Poder Judicia.

\subsection{Número de estaciones de Transmisión FM}

\begin{tabular}{|r|r|} \hline
	Canalización 			& 28 \\ \hline
	Autorizaciones			& 25 \\ \hline
	Cautelar				& 0 \\ \hline
	Reservado Estado		& 1 \\ \hline
	Disponibilidad			& 2 \\ \hline
	Trámite					& 1 \\ \hline 
\end{tabular}

\subsection{Límite de potencia de transmisión}

La Potencia Efectiva Radiada máxima en la dirección de máxima ganancia de antena a ser autorizada en esta localidad será: 1 KW.

\subsection{Límites geográficos de autorización}

La transmisioón está autorizada para llegar hasta la provincia de Canchis, el distrito/centro poblado:

\begin{itemize}
	\item Combapata
	\item Marangani
	\item San Pablo
	\item San Pedro
	\item Sicuani
	\item Tinta
\end{itemize}



\section{Chumbivilcas - Santo Tomas}

\begin{itemize}
	\item Plan Aprobado por RVM Nº 426-2014-MTC/03 (28/07/2014)
	\item Con Restricción RD Nº 1676-2017-MTC/28 (18/08/2017)
\end{itemize}

El Registro Nacional de Frecuencias incluye como "autorizadas" a las estaciones con autorización vigente o que se encuentran en proceso de renovación, y no cuentan con resolución que la deje sin efecto o extinción firme. El estado “cautelar” corresponde a estaciones cuya autorización ha sido cancelada en la vía administrativa, pero cuentan con una medida cautelar otorgada por el Poder Judicia.

\subsection{Número de estaciones de Transmisión FM}

\begin{tabular}{|r|r|} \hline
	Canalización 			& 20 \\ \hline
	Autorizaciones			& 17 \\ \hline
	Cautelar				& 0 \\ \hline
	Reservado Estado		& 1 \\ \hline
	Disponibilidad			& 2 \\ \hline
	Trámite					& 0 \\ \hline 
\end{tabular}

\subsection{Límite de potencia de transmisión}

La Potencia Efectiva Radiada máxima en la dirección de máxima ganancia de antena a ser autorizada en esta localidad será: 0.25 KW.

\subsection{Límites geográficos de autorización}

La transmisioón está autorizada para llegar hasta la provincia de Chumbivilcas, el distrito/centro poblado:

\begin{itemize}
	\item Santo Tomas
\end{itemize}



\section{Cusco - Cusco}

\begin{itemize}
	\item Plan Aprobado por RVM Nº 108-2004-MTC/03 (10/07/2004)
	\item Con Restricción RD Nº 1164-2006-MTC/17 (28/06/2006)
	\item Concurso Público Nº 01-2015-MTC/28 (30/12/2015), RD Nº 2129-2015-MTC/28 (29/12/2015)
\end{itemize}

El Registro Nacional de Frecuencias incluye como "autorizadas" a las estaciones con autorización vigente o que se encuentran en proceso de renovación, y no cuentan con resolución que la deje sin efecto o extinción firme. El estado “cautelar” corresponde a estaciones cuya autorización ha sido cancelada en la vía administrativa, pero cuentan con una medida cautelar otorgada por el Poder Judicia.

\subsection{Número de estaciones de Transmisión FM}

\begin{tabular}{|r|r|} \hline
	Canalización 			& 28 \\ \hline
	Autorizaciones			& 24 \\ \hline
	Cautelar				& 0 \\ \hline
	Reservado Estado		& 1 \\ \hline
	Disponibilidad			& 3 \\ \hline
	Trámite					& 1 \\ \hline 
\end{tabular}

\subsection{Límite de potencia de transmisión}

La Potencia Efectiva Radiada máxima en la dirección de máxima ganancia de antena a ser autorizada en esta localidad será: 1 KW.

\subsection{Límites geográficos de autorización}

La transmisioón está autorizada para llegar hasta la provincia de Anta y Cusco, el distrito/centro poblado:

\begin{itemize}
	\item Cachimayo
	\item Cusco
	\item Poroy
	\item Poroy
	\item San Jeronimo
	\item San Sebastian
	\item Santiago
	\item Saylla
	\item Wanchaq
\end{itemize}



\section{Espinar - Espinar}

\begin{itemize}
	\item Plan Aprobado por RVM Nº 426-2014-MTC/03 (28/07/2014)
	\item Con Restricción RD Nº 1135-2012-MTC/28 (26/09/2012)
	\item Concurso Público Nº CP 02-2013-MTC/28 (18/12/2013), RD Nº 1841-2013-MTC/28 (12/12/2013)
\end{itemize}

El Registro Nacional de Frecuencias incluye como "autorizadas" a las estaciones con autorización vigente o que se encuentran en proceso de renovación, y no cuentan con resolución que la deje sin efecto o extinción firme. El estado “cautelar” corresponde a estaciones cuya autorización ha sido cancelada en la vía administrativa, pero cuentan con una medida cautelar otorgada por el Poder Judicia.

\subsection{Número de estaciones de Transmisión FM}

\begin{tabular}{|r|r|} \hline
	Canalización 			& 29 \\ \hline
	Autorizaciones			& 22 \\ \hline
	Cautelar				& 0 \\ \hline
	Reservado Estado		& 2 \\ \hline
	Disponibilidad			& 5 \\ \hline
	Trámite					& 0 \\ \hline 
\end{tabular}

\subsection{Límite de potencia de transmisión}

La Potencia Efectiva Radiada máxima en la dirección de máxima ganancia de antena a ser autorizada en esta localidad será: 1 KW.

\subsection{Límites geográficos de autorización}

La transmisioón está autorizada para llegar hasta la provincia de Espinar, el distrito/centro poblado:

\begin{itemize}
	\item Alto Pichigua
	\item Coporaque
	\item Espinar
	\item Pichigua
\end{itemize}



\section{La convención - Santa Ana}

\begin{itemize}
	\item Plan Aprobado por RVM Nº 108-2004-MTC/03 (10/07/2004)
	\item Con Restricción RD Nº 3548-2010-MTC/28 (12/11/2010)
	\item Concurso Público Nº 02-2019-MTC/28 (23/09/2019), RD Nº 2770-2019-MTC/28 (20/09/2019)
\end{itemize}

El Registro Nacional de Frecuencias incluye como "autorizadas" a las estaciones con autorización vigente o que se encuentran en proceso de renovación, y no cuentan con resolución que la deje sin efecto o extinción firme. El estado “cautelar” corresponde a estaciones cuya autorización ha sido cancelada en la vía administrativa, pero cuentan con una medida cautelar otorgada por el Poder Judicia.

\subsection{Número de estaciones de Transmisión FM}

\begin{tabular}{|r|r|} \hline
	Canalización 			& 28 \\ \hline
	Autorizaciones			& 19 \\ \hline
	Cautelar				& 0 \\ \hline
	Reservado Estado		& 1 \\ \hline
	Disponibilidad			& 8 \\ \hline
	Trámite					& 1 \\ \hline 
\end{tabular}

\subsection{Límite de potencia de transmisión}

La Potencia Efectiva Radiada máxima en la dirección de máxima ganancia de antena a ser autorizada en esta localidad será: 1 KW.

\subsection{Límites geográficos de autorización}

La transmisioón está autorizada para llegar hasta la provincia de La Convención, el distrito/centro poblado:

\begin{itemize}
	\item Maranura
	\item Santa Ana
\end{itemize}



\section{Paruro - Paruro}

\begin{itemize}
	\item Plan Aprobado por RVM Nº 426-2014-MTC/03 (28/07/2014)
	\item Con Restricción RD Nº 0265-2017-MTC/28 (15/03/2017)
\end{itemize}

El Registro Nacional de Frecuencias incluye como "autorizadas" a las estaciones con autorización vigente o que se encuentran en proceso de renovación, y no cuentan con resolución que la deje sin efecto o extinción firme. El estado “cautelar” corresponde a estaciones cuya autorización ha sido cancelada en la vía administrativa, pero cuentan con una medida cautelar otorgada por el Poder Judicia.

\subsection{Número de estaciones de Transmisión FM}

\begin{tabular}{|r|r|} \hline
	Canalización 			& 16 \\ \hline
	Autorizaciones			& 8 \\ \hline
	Cautelar				& 0 \\ \hline
	Reservado Estado		& 2 \\ \hline
	Disponibilidad			& 6 \\ \hline
	Trámite					& 0 \\ \hline 
\end{tabular}

\subsection{Límite de potencia de transmisión}

La Potencia Efectiva Radiada máxima en la dirección de máxima ganancia de antena a ser autorizada en esta localidad será: 0.5 KW.

\subsection{Límites geográficos de autorización}

La transmisioón está autorizada para llegar hasta la provincia de Paruro, el distrito/centro poblado:

\begin{itemize}
	\item Rondocan
	\item Colcha
	\item Paruro
\end{itemize}



\section{Paucartambo - Paucartambo}

\begin{itemize}
	\item Plan Aprobado por RVM Nº 426-2014-MTC/03 (28/07/2014)
	\item Con Restricción RD Nº 1953-2016-MTC/28 (17/12/2016)
\end{itemize}

El Registro Nacional de Frecuencias incluye como "autorizadas" a las estaciones con autorización vigente o que se encuentran en proceso de renovación, y no cuentan con resolución que la deje sin efecto o extinción firme. El estado “cautelar” corresponde a estaciones cuya autorización ha sido cancelada en la vía administrativa, pero cuentan con una medida cautelar otorgada por el Poder Judicia.

\subsection{Número de estaciones de Transmisión FM}

\begin{tabular}{|r|r|} \hline
	Canalización 			& 15 \\ \hline
	Autorizaciones			& 8 \\ \hline
	Cautelar				& 0 \\ \hline
	Reservado Estado		& 0 \\ \hline
	Disponibilidad			& 7 \\ \hline
	Trámite					& 0 \\ \hline 
\end{tabular}

\subsection{Límite de potencia de transmisión}

La Potencia Efectiva Radiada máxima en la dirección de máxima ganancia de antena a ser autorizada en esta localidad será: 0.5 KW.

\subsection{Límites geográficos de autorización}

La transmisioón está autorizada para llegar hasta la provincia de Paucartambo, el distrito/centro poblado:

\begin{itemize}
	\item Challabamba
	\item Colquepata
	\item Paucartambo
\end{itemize}



\section{Quispicanchi - Urcos}

\begin{itemize}
	\item Plan Aprobado por RVM Nº 1141-2016-MTC/03 (27/07/2016)
	\item Con Restricción RD Nº 1776-2012-MTC/28 (18/01/2013)
	\item Concurso Público Nº 01-2015-MTC/28 (30/12/2015), RD Nº 2129-2015-MTC/28 (29/12/2015)
\end{itemize}

El Registro Nacional de Frecuencias incluye como "autorizadas" a las estaciones con autorización vigente o que se encuentran en proceso de renovación, y no cuentan con resolución que la deje sin efecto o extinción firme. El estado “cautelar” corresponde a estaciones cuya autorización ha sido cancelada en la vía administrativa, pero cuentan con una medida cautelar otorgada por el Poder Judicia.

\subsection{Número de estaciones de Transmisión FM}

\begin{tabular}{|r|r|} \hline
	Canalización 			& 26 \\ \hline
	Autorizaciones			& 21 \\ \hline
	Cautelar				& 0 \\ \hline
	Reservado Estado		& 2 \\ \hline
	Disponibilidad			& 3 \\ \hline
	Trámite					& 0 \\ \hline 
\end{tabular}

\subsection{Límite de potencia de transmisión}

La Potencia Efectiva Radiada máxima en la dirección de máxima ganancia de antena a ser autorizada en esta localidad será: 0.5 KW.

\subsection{Límites geográficos de autorización}

La transmisioón está autorizada para llegar hasta la provincia de Paucartambo y Quispicanchi, el distrito/centro poblado:

\begin{itemize}
	\item Caicay
	\item Andahuaylillas
	\item Huaro
	\item Lucre
	\item Oropesa
	\item Urcos
\end{itemize}



\section{Urubamba - Urubamba}

\begin{itemize}
	\item Plan Aprobado por RVM Nº 108-2004-MTC/03 (10/07/2004)
	\item Con Restricción RD Nº 2189-2005-MTC/17 (12/01/2006)
	\item Concurso Público Nº 02-2012-MTC/28 (17/11/2012), RD Nº 1555-2012-MTC/28 (07/11/2012)
\end{itemize}

El Registro Nacional de Frecuencias incluye como "autorizadas" a las estaciones con autorización vigente o que se encuentran en proceso de renovación, y no cuentan con resolución que la deje sin efecto o extinción firme. El estado “cautelar” corresponde a estaciones cuya autorización ha sido cancelada en la vía administrativa, pero cuentan con una medida cautelar otorgada por el Poder Judicia.

\subsection{Número de estaciones de Transmisión FM}

\begin{tabular}{|r|r|} \hline
	Canalización 			& 28 \\ \hline
	Autorizaciones			& 24 \\ \hline
	Cautelar				& 0 \\ \hline
	Reservado Estado		& 2 \\ \hline
	Disponibilidad			& 2 \\ \hline
	Trámite					& 1 \\ \hline 
\end{tabular}

\subsection{Límite de potencia de transmisión}

La Potencia Efectiva Radiada máxima en la dirección de máxima ganancia de antena a ser autorizada en esta localidad será: 0.5 KW.

\subsection{Límites geográficos de autorización}

La transmisioón está autorizada para llegar hasta la provincia de Anta y Urubamba, el distrito/centro poblado:

\begin{itemize}
	\item Arcahuasi
	\item Anta
	\item Huarocondo
	\item Pucyura
	\item Zurite
	\item Chinchero
	\item Huayllabamba
	\item Maras
	\item Ollantaytambo
	\item Urubamba
	\item Yucay
\end{itemize}


\section{Procedimiento para obtener autorización de emisión en radiodifusión}

\subsection{Conceptos Generales}
Es un servicio de telecomunicaciones, cuya 
programación es recibida por el público en 
general. 

Se clasifica en 

\begin{itemize}
	\item Por la modalidad de operación
	\begin{itemize}
		\item Sonora: a su vez se clasifica en: 
		Frecuencia Modulada, Onda Media y 
		Onda Corta (Tropical e Internacional).
		\item Por televisión: a  su  vez  se  clasifica  en: 
		VHF y UHF
	\end{itemize}
	\item Por la finalidad:
	\begin{itemize}
		\item Comercial
		\item Educativa
		\item Comunitaria
	\end{itemize}
\end{itemize}

\subsubsection{Radiodifusión Comercial}
La Radiodifusión es aquella cuyos programas contienen temas 
relacionados  con  el  entretenimiento,  recreación, 
información,  noticiosos  y  de  orientación  de  la 
comunidad.

\subsubsection{Radiodifusión Comercial}
Es aquella cuyos programas contienen temas 
relacionados,  principalmente,  con  la  educación, 
cultura y deporte. 

\subsubsection{Radiodifusión Educativa}

Es aquella cuyos programas contienen temas 
relacionados, principalmente, con el fomento de 
la identidad, costumbres de la comunidad e 
integración nacional. La estación radiodifusora  debe necesariamente 
ubicarse  en  comunidades  campesinas,  nativas  e 
indígenas,  áreas  rurales  o  de  preferente  interés 
social. 

\subsection{Procedimiento}

Cualquier persona sea natural o persona jurídica puede prestar el servicio . Para la prestación del servicio de radiodifusión se 
requiere  contar  con  una  autorización  otorgada 
por resolución del Viceministro de 
Comunicaciones.

Para la prestación de servicio se necesita una autorización que es la que faculta la prestación de un servicio de radiodifusión. La solicitud se presenta 
ante  la  Dirección  General  de  Autorizaciones  en 
Telecomunicaciones.  Para  tal  efecto  se  deberá 
emplear los formatos que están disponibles en la 
página web del MTC (www.mtc.gob.pe)

Las autorizaciones se otorgan por un plazo 
máximo de diez (10) años. 

\subsubsection{Procedimientos están sujetos al silencio  administrativo positivo}

\begin{itemize}
	\item Transferencia de autorización. 
	\item Renovación de autorización. 
	\item Transferencia de acciones, participaciones, 
	titularidad y modificación del representante 
	legal, directorio o consejo directivo.
\end{itemize}

\subsubsection{Procedimientos están sujetos al silencio administrativo negativo}
\begin{itemize}
	\item Autorización para prestar el servicio de 
	radiodifusión.
	\item Modificación de características técnicas 
	(aumento de potencia y cambio de ubicación de 
	la  planta  transmisora)  y  de  condición  esencial 
	(frecuencia o canal asignado). 
	\item Otorgamiento de pago fraccionado.
	\item Otorgamiento de nuevo pago fraccionado.
\end{itemize}

\subsubsection{Las modalidades de otorgamiento de una Autorización}
\begin{enumerate}
	\item A solicitud de Parte
	\item Por Concurso público
\end{enumerate}

El otorgamiento de 
autorización a solicitud de parte procede cuando el número de frecuencias o canales 
disponibles  en  una  misma  banda  y  localidad  es 
superior a las solicitudes admitidas.

Una Autorización por concurso público se otorga cuando existe restricción en la disponibilidad de 
frecuencias  o  canales  en  una  misma  banda  y 
localidad. 

La restricción en la disponibilidad de frecuencias o canales es  cuando  el  número  de  frecuencias  o  canales 
disponibles  es  inferior  al  número  de  solicitudes 
admitidas.

Las frecuencias o 
canales que pueden autorizarse en una 
determinada localidad se  enumeran  en  los  Planes  de  Canalización  y 
Asignación de Frecuencias, allí se indican el 
máximo de frecuencias o canales que pueden ser 
asignados en una localidad. 
	
\end{document}
